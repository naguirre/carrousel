\documentclass{beamer}
\usetheme{Warsaw}
\usepackage{tikz}
\usetikzlibrary{shapes.callouts,shadows, calc}
\usepackage{thumbpdf}
\usepackage{wasysym}
\usepackage{ucs}
\usepackage[utf8]{inputenc}
\usepackage{pgf,pgfarrows,pgfnodes,pgfautomata,pgfheaps,pgfshade}
\usepackage{verbatim}
\usepackage{listings}
\usepackage{color}

\definecolor{dkgreen}{rgb}{0,0.6,0}
\definecolor{gray}{rgb}{0.5,0.5,0.5}
\definecolor{mauve}{rgb}{0.58,0,0.82}

\lstset{ %
  backgroundcolor=\color{white},  % choose the background color; you must add \usepackage{color} or \usepackage{xcolor}
  basicstyle=\footnotesize,       % the size of the fonts that are used for the code
  breakatwhitespace=false,        % sets if automatic breaks should only happen at whitespace
  breaklines=true,                % sets automatic line breaking
  captionpos=b,                   % sets the caption-position to bottom
  commentstyle=\color{dkgreen},   % comment style
  deletekeywords={...},           % if you want to delete keywords from the given language
  escapeinside={\%*}{*)},         % if you want to add LaTeX within your code
  extendedchar=true,              % lets you use non-ASCII characters; for 8-bits encodings only, does not work with UTF-8
  frame=single,                   % adds a frame around the code
  keywordstyle=\color{blue},      % keyword style
  language=Octave,                % the language of the code
  morekeywords={*,...},           % if you want to add more keywords to the set
  numbers=left,                   % where to put the line-numbers; possible values are (none, left, right)
  numbersep=5pt,                  % how far the line-numbers are from the code
  numberstyle=\tiny\color{gray},  % the style that is used for the line-numbers
  rulecolor=\color{black},        % if not set, the frame-color may be changed on line-breaks within not-black text (e.g. comments (green here))
  showspaces=false,               % show spaces everywhere adding particular underscores; it overrides 'showstringspaces'
  showstringspaces=false,         % underline spaces within strings only
  showtabs=false,                 % show tabs within strings adding particular underscores
  stepnumber=2,                   % the step between two line-numbers. If it's 1, each line will be numbered
  stringstyle=\color{mauve},      % string literal style
  tabsize=2,                      % sets default tabsize to 2 spaces
  title=\lstname                  % show the filename of files included with \lstinputlisting; also try caption instead of title
}

\tikzset{note/.style={rectangle callout, rounded corners,fill=gray!20,drop shadow,font=\footnotesize}}    

\newcommand{\tikzmark}[1]{\tikz[overlay,remember picture] \node (#1) {};}    

\newcounter{image}
\setcounter{image}{1}

\makeatletter
\newenvironment{btHighlight}[1][]
{\begingroup\tikzset{bt@Highlight@par/.style={#1}}\begin{lrbox}{\@tempboxa}}
{\end{lrbox}\bt@HL@box[bt@Highlight@par]{\@tempboxa}\endgroup}

\newcommand\btHL[1][]{%
  \begin{btHighlight}[#1]\bgroup\aftergroup\bt@HL@endenv%
}
\def\bt@HL@endenv{%
  \end{btHighlight}%   
  \egroup
}
\newcommand{\bt@HL@box}[2][]{%
  \tikz[#1]{%
    \pgfpathrectangle{\pgfpoint{0pt}{0pt}}{\pgfpoint{\wd #2}{\ht #2}}%
    \pgfusepath{use as bounding box}%
    \node[anchor=base west,rounded corners, fill=green!30,outer sep=0pt,inner xsep=0.2em, inner ysep=0.1em,  #1](a\theimage){\usebox{#2}};
  }%
   %\tikzmark{a\theimage} <= can be used, but it leads to a spacing problem
   % the best approach is to name the previous node with (a\theimage)
 \stepcounter{image}
}
\makeatother


\lstset{language=C,
  basicstyle=\footnotesize\ttfamily,
  keywordstyle=\footnotesize\color{blue}\ttfamily,
  moredelim=**[is][\btHL]{`}{`},
}

\begin{document}

\pdfinfo
{
  /Title       (Un carrousel en EFL)
  /Creator     (Tex)
  /Author      (Nicolas Aguirre)
}


\title{Un carrousel en EFL}
\subtitle{Edje, Evas et Elementary sont dans un manège}
\author{Nicolas Aguirre}
\date{26 Janvier 2013}

\frame{\titlepage}

\section{Introduction}
\begin{frame}
  \frametitle{Notions que nous aborderons}
  \begin{block}{Edje}
  \begin{itemize}
    \item<2-> Création d'un fichier edje
    \item<3-> Utilisation d'un fichier edje dans elementary
    \item<4-> Création de groupes edje et utilisation en tant qu'objets Evas
  \end{itemize}
  \end{block}

  \begin{block}{Elementary}
  \begin{itemize}
    \item<5-> Création d'une fenêtre
    \item<6-> Intégration d'un layout edje dans la fenêtre
  \end{itemize}
  \end{block}

  \begin{block}{Evas}
  \begin{itemize}
    \item<7-> Création d'un Objet Evas
    \item<8-> Manipulation d'objets Evas
  \end{itemize}
  \end{block}

\end{frame}

\begin{frame}
  \frametitle{Show me the code !}
  Tout le Code de ce tutoriel est sous licence GPLv3
  Il est disponible à cette adresse :
  \begin{block}{Utilisez git ou allez en enfer !}
    git clone https://github.com/naguirre/carrousel.git
  \end{block}
  \lstset{language=C,caption={Descriptive Caption Text},label=DescriptiveLabel}
\end{frame}

\section{Création d'une fenêtre}
\begin{frame}[fragile]{Hello World}

Cette étape permet la mise en place des autotools et du fichier main.c\\
Le fichier configure.ac contient en check sur elementary
\begin{lstlisting}
PKG_CHECK_MODULES([CARROUSEL], [elementary >= 1.0.0])
\end{lstlisting}
Le fichier Makefile.am contient les fichier a compiler, uniquement main.c pour le moment.\\
Et voici le contenu du fichier main.c :
\begin{lstlisting}
#include <Elementary.h>

int main(int argc, char **argv)
{
    printf("Hello E World\n");
    return 0;
}
\end{lstlisting}
\end{frame}


\begin{frame}[fragile]{Récupération du code et Compilation}
\begin{block}{Récupération du code :}
git checkout step1
\end{block}
\begin{block}{Compilation tres classique avec}
\begin{verbatim}
./autogen.sh
./configure
make
make install
\end{verbatim}
\end{block}
\end{frame}

\begin{frame}[fragile]{Step2 : ELementarization}
\begin{lstlisting}
#include <Elementary.h>

#ifndef ELM_LIB_QUICKLAUNCH

EAPI_MAIN int
elm_main(int argc EINA_UNUSED, char **argv EINA_UNUSED)
{
    printf("Hello Elementary World\n");
    return 0;
}

#endif

ELM_MAIN()
\end{lstlisting}

\end{frame}

\begin{frame}[fragile]{Step3}
Création de la fenêtre :
\begin{lstlisting}
  win = elm_win_add(NULL, "main", ELM_WIN_BASIC);
  elm_win_title_set(win, "EFL Demo");
  ....
  evas_object_resize(win, 800, 600);
  evas_object_show(win);
\end{lstlisting}

Création d'un fond :
\begin{lstlisting}
  bg = elm_bg_add(win);
  elm_win_resize_object_add(win, bg);
  evas_object_show(bg);
\end{lstlisting}


Boucle de messages :
\begin{lstlisting}
    elm_run();
\end{lstlisting}
\end{frame}

\begin{frame}[fragile]{Step4}
Fermeture de la fenêtre :
\begin{lstlisting}
 evas_object_smart_callback_add(win, "delete,request", _win_del, NULL);
\end{lstlisting}

Callback de fermeture :
\begin{lstlisting}
static void
_win_del(void *data EINA_UNUSED, Evas_Object *obj EINA_UNUSED, void *event_info EINA_UNUSED)
{
    elm_exit();
}
\end{lstlisting}
\end{frame}

\section{Création d'un layout edje}
\begin{frame}[fragile]{Step4}
Création d'un fichier edje contenant un group layout
\begin{lstlisting}
   group {
      name: "layout";
      parts {
         part {
            name: "bg";
            type: IMAGE;
            description { image.normal: "bg.png"; }
         }
         part {
            name: "caroussel.swallow";
            type: SWALLOW;
            mouse_events: 1;
            description { state: "default" 0.0; }
         }
      }
   }
\end{lstlisting}
\end{frame}

\begin{frame}[fragile]{Intégration du layout}
On créé un nouveau layout elementary. On charge le fichier edje précédement créé et on charge le groupe ``layout''.
On en profite également pour faire en sorte que la dimension de la fenêtre soit liée a celle de l'object layout.
Et finalement on affiche le layout à l'écran.\\
Un layout elementary est un frontend a edje, permettant de charger des fichiers et des groupes Edje.\\
Au lieu de manipuler un object edje, on manipule un object elementary.\\
Dans les deux cas se sont des Evas\_Objects.\\
Une préférence va a l'utilisation des objets elementary, car il héritent des propriétés globales de elm (theme, finger size ...)
\end{frame}

\begin{frame}[fragile]{code}
\begin{lstlisting}
  ly = elm_layout_add(win);
  elm_layout_file_set(ly,
        PACKAGE_DATA_DIR"/themes/default/default.edj",
        "layout");
  elm_win_resize_object_add(win, ly);
  evas_object_show(ly);
\end{lstlisting}
\end{frame}

\section{Création d'un objet carrousel}
\begin{frame}[fragile]{Step5}
Création d'un object carrousel basé sur un elm\_grid. (carrousel.[ch])
\begin{lstlisting}
Evas_Object *
carrousel_add(Evas_Object *parent)
{
    Evas_Object *grid;

    grid =  elm_grid_add(parent);
    evas_object_grid_size_set(grid, 800, 600);

    return grid;
}
\end{lstlisting}
Intégration de l'objet dans l'interface
\begin{lstlisting}
elm_object_part_content_set(ly, "caroussel.swallow", carrousel);
\end{lstlisting}
\end{frame}

\begin{frame}[fragile]{Elm\_Grid}
On spécifie une taille de grille de 800x600px. \\
Une elm grid permet de placer librement des Evas Objects a l'intérieur. Les objets insérés sont alors des enfants de la grille. Lorsque la grille est supprimée, les enfants sont supprimés a leur tour.\\
On pourrait utiliser evas\_object\_move et evas\_object\_resize en lieu et place de elm\_grid.\\
Elm\_Grid permet cependant de gérer l'arbre d'objets aisi que l'héritage des paramétres génériques ELM.
Pour placer des objets dans la grille on utilise :
\begin{lstlisting}
elm_grid_pack(grid, item->obj, 0, 0, 0, 0);
\end{lstlisting}
\end{frame}

\section{Création des éléments du carrousel}

\begin{frame}[fragile]{Step6-Step7}
Les éléments du carrousel sont des elm\_layouts basés sur un groupe edje : ``caroussel/layout/item''.
On créé dans un premier temps 8 Objets que l'on place a l'écran. L'affichage a l'écran se fait dans la fonction \_anim(). Tous les objets créés sont ajoutés dans un Eina List.
\end{frame}
\begin{frame}[fragile]{la fonction \_anim}
\begin{lstlisting}
static Eina_Bool
_anim(void)
{
    Carrousel_Item *item;
    Eina_List *l;
    Evas_Coord x, y, w, h;

    EINA_LIST_FOREACH(items, l, item)
    {
        y = (HEIGHT / 2) - (ICON_SIZE_H / 2);
        x = (WIDTH  / 2) - (ICON_SIZE_W / 2);
        w = ICON_SIZE_W;
        h = ICON_SIZE_H;
        elm_grid_pack_set(item->obj, x, y, w, h);
    }
    return ECORE_CALLBACK_RENEW;
}
\end{lstlisting}
\end{frame}

\begin{frame}[fragile]{avec du padding}
\begin{lstlisting}
static Eina_Bool
_anim(void)
{
    Carrousel_Item *item;
    Eina_List *l;
    Evas_Coord x, y, w, h;
    int i = 0;
    EINA_LIST_FOREACH(items, l, item)
    {
        y = HEIGHT / 2 - ICON_SIZE_H / 2;
        x = PADDING + i * (ICON_SIZE_W + PADDING);
        w = ICON_SIZE_W;
        h = ICON_SIZE_H;
        elm_grid_pack_set(item->obj, x, y, w, h);
        i++;
    }
    return ECORE_CALLBACK_RENEW;
}
\end{lstlisting}
\end{frame}

\begin{frame}[fragile]{Chargement des images sur le disque}
Ajoutons des images dans les layouts. \\
Utilisation de evas\_object\_image
\begin{lstlisting}
snprintf(buf, sizeof(buf), PACKAGE_DATA_DIR"/images/%s",  files[i % 7]);
img = evas_object_image_filled_add(evas_object_evas_get(item->obj));
evas_object_image_file_set(img, buf, NULL);
elm_object_part_content_set(item->obj, "cover.swallow", img);
\end{lstlisting}
\end{frame}

\section{Ajoutons de la pseudo 3D}
\begin{frame}[fragile]{Rappel trigo}

\end{frame}

\end{document}
