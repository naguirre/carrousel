\documentclass[12pt]{beamer}

\usetheme{Warsaw}
\usepackage{thumbpdf}
\usepackage{wasysym}
\usepackage{ucs}
\usepackage[utf8]{inputenc}
\usepackage{pgf,pgfarrows,pgfnodes,pgfautomata,pgfheaps,pgfshade}
\usepackage{verbatim}
\usepackage{listings}

\pdfinfo
{
  /Title       (Un carrousel en EFL)
  /Creator     (Tex)
  /Author      (Nicolas Aguirre)
}


\title{Un carrousel en EFL}
\subtitle{Edje, Evas et Elementary sont dans un manège}
\author{Nicolas Aguirre}
\date{26 Janvier 2013}

\begin{document}

\frame{\titlepage}

\section{Introduction}
\begin{frame}
  \frametitle{Notions que nous aborderons}
  \begin{block}{Edje}
  \begin{itemize}
    \item<2-> Création d'un fichier edje
    \item<3-> Utilisation d'un fichier edje dans elementary
    \item<4-> Création de groupes edje et utilisation en tant qu'objets Evas
  \end{itemize}
  \end{block}

  \begin{block}{Elementary}
  \begin{itemize}
    \item<5-> Création d'une fenêtre
    \item<6-> Intégration d'un layout edje dans la fenêtre
  \end{itemize}
  \end{block}

  \begin{block}{Evas}
  \begin{itemize}
    \item<7-> Création d'un Objet Evas
    \item<8-> Manipulation d'objets Evas
  \end{itemize}
  \end{block}

\end{frame}

\begin{frame}
  \frametitle{Show me the code !}
  Tout le Code de ce tutoriel est sous licence GPLv3
  Il est disponible à cette adresse :
  \begin{block}{Utilisez git ou allez en enfer !}
    git clone https://github.com/naguirre/carrousel.git
  \end{block}
\end{frame}

\section{Creer un fenêtre}

\begin{frame}
  \frametitle{git checkout step1}
  \begin{lstlisting}
 
  \end{lstlisting}
\end{frame}

\end{document}
